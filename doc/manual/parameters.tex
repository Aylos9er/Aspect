\subsection{Global parameters}
\label{parameters:global}


\begin{itemize}
\item {\it Parameter name:} {\tt Adiabatic surface temperature}


\index[prmindex]{Adiabatic surface temperature}
\index[prmindexfull]{Adiabatic surface temperature}
{\it Value:} 0


{\it Default:} 0


{\it Description:} In order to make the problem in the first time step easier to solve, we need a reasonable guess for the temperature and pressure. To obtain it, we use an adiabatic pressure and temperature field. This parameter describes what the `adiabatic' temperature would be at the surface of the domain (i.e. at depth zero). Note that this value need not coincide with the boundary condition posed at this point. Rather, the boundary condition may differ significantly from the adiabatic value, and then typically induce a thermal boundary layer.
For more information, see the section in the manual that discusses the general mathematical model.


{\it Possible values:} [Double -1.79769e+308...1.79769e+308 (inclusive)]
\item {\it Parameter name:} {\tt CFL number}


\index[prmindex]{CFL number}
\index[prmindexfull]{CFL number}
{\it Value:} 1.0


{\it Default:} 1.0


{\it Description:} In computations, the time step $k$ is chosen according to $k = c \min_K \frac{h_K}{\|u\|_{\infty,K} p_T}$ where $h_K$ is the diameter of cell $K$, and the denominator is the maximal magnitude of the velocity on cell $K$ times the polynomial degree $p_T$ of the temperature discretization. The dimensionless constant $c$ is called the CFL number in this program. For time discretizations that have explicit components, $c$ must be less than a constant that depends on the details of the time discretization and that is no larger than one. On the other hand, for implicit discretizations such as the one chosen here, one can choose the time step as large as one wants (in particular, one can choose $c>1$) though a CFL number significantly larger than one will yield rather diffusive solutions. Units: None.


{\it Possible values:} [Double 0...1.79769e+308 (inclusive)]
\item {\it Parameter name:} {\tt End time}


\index[prmindex]{End time}
\index[prmindexfull]{End time}
{\it Value:} 1e10


{\it Default:} 1e8


{\it Description:} The end time of the simulation. Units: years if the 'Use years in output instead of seconds' parameter is set; seconds otherwise.


{\it Possible values:} [Double -1.79769e+308...1.79769e+308 (inclusive)]
\item {\it Parameter name:} {\tt Output directory}


\index[prmindex]{Output directory}
\index[prmindexfull]{Output directory}
{\it Value:} output


{\it Default:} output


{\it Description:} The name of the directory into which all output files should be placed. This may be an absolute or a relative path.


{\it Possible values:} [DirectoryName]
\item {\it Parameter name:} {\tt Resume computation}


\index[prmindex]{Resume computation}
\index[prmindexfull]{Resume computation}
{\it Value:} false


{\it Default:} false


{\it Description:} A flag indicating whether the computation should be resumed from a previously saved state (if true) or start from scratch (if false).


{\it Possible values:} [Bool]
\item {\it Parameter name:} {\tt Start time}


\index[prmindex]{Start time}
\index[prmindexfull]{Start time}
{\it Value:} 0


{\it Default:} 0


{\it Description:} The start time of the simulation. Units: years if the 'Use years in output instead of seconds' parameter is set; seconds otherwise.


{\it Possible values:} [Double -1.79769e+308...1.79769e+308 (inclusive)]
\item {\it Parameter name:} {\tt Surface pressure}


\index[prmindex]{Surface pressure}
\index[prmindexfull]{Surface pressure}
{\it Value:} 0


{\it Default:} 0


{\it Description:} The mathematical equations that describe thermal convection only determine the pressure up to an arbitrary constant. On the other hand, for comparison and for looking up material parameters it is important that the pressure be normalized somehow. We do this by enforcing a particular average pressure value at the surface of the domain, where the geometry model determines where the surface is. This parameter describes what this average surface pressure value is supposed to be. By default, it is set to zero, but one may want to choose a different value for example for simulating only the volume of the mantle below the lithosphere, in which case the surface pressure should be the lithostatic pressure at the bottom of the lithosphere.
For more information, see the section in the manual that discusses the general mathematical model.


{\it Possible values:} [Double -1.79769e+308...1.79769e+308 (inclusive)]
\item {\it Parameter name:} {\tt Use years in output instead of seconds}


\index[prmindex]{Use years in output instead of seconds}
\index[prmindexfull]{Use years in output instead of seconds}
{\it Value:} false


{\it Default:} true


{\it Description:} When computing results for mantle convection simulations, it is often difficult to judge the order of magnitude of results when they are stated in MKS units involving seconds. Rather, some kinds of results such as velocities are often stated in terms of meters per year (or, sometimes, centimeters per year). On the other hand, for non-dimensional computations, one wants results in their natural unit system as used inside the code. If this flag is set to 'true' conversion to years happens; if it is 'false', no such conversion happens.


{\it Possible values:} [Bool]
\end{itemize}



\subsection{Parameters in section \tt Boundary temperature model}
\label{parameters:Boundary_20temperature_20model}

\begin{itemize}
\item {\it Parameter name:} {\tt Model name}


\index[prmindex]{Model name}
\index[prmindexfull]{Boundary temperature model!Model name}
{\it Value:} box


{\it Default:}


{\it Description:} Select one of the following models:

`spherical constant': A model in which the temperature is chosen constant on the inner and outer boundaries of a spherical shell. Parameters are read from subsection 'Sherical constant'.

`box': A model in which the temperature is chosen constant on the left and right sides of a box.


{\it Possible values:} [Selection spherical constant|box ]
\end{itemize}



\subsection{Parameters in section \tt Boundary temperature model/Spherical constant}
\label{parameters:Boundary_20temperature_20model/Spherical_20constant}

\begin{itemize}
\item {\it Parameter name:} {\tt Inner temperature}


\index[prmindex]{Inner temperature}
\index[prmindexfull]{Boundary temperature model!Spherical constant!Inner temperature}
{\it Value:} 6000


{\it Default:} 6000


{\it Description:} Temperature at the inner boundary (core mantle boundary). Units: K.


{\it Possible values:} [Double -1.79769e+308...1.79769e+308 (inclusive)]
\item {\it Parameter name:} {\tt Outer temperature}


\index[prmindex]{Outer temperature}
\index[prmindexfull]{Boundary temperature model!Spherical constant!Outer temperature}
{\it Value:} 0


{\it Default:} 0


{\it Description:} Temperature at the outer boundary (lithosphere water/air). Units: K.


{\it Possible values:} [Double -1.79769e+308...1.79769e+308 (inclusive)]
\end{itemize}

\subsection{Parameters in section \tt Discretization}
\label{parameters:Discretization}

\begin{itemize}
\item {\it Parameter name:} {\tt Stokes velocity polynomial degree}


\index[prmindex]{Stokes velocity polynomial degree}
\index[prmindexfull]{Discretization!Stokes velocity polynomial degree}
{\it Value:} 2


{\it Default:} 2


{\it Description:} The polynomial degree to use for the velocity variables in the Stokes system. Units: None.


{\it Possible values:} [Integer range 1...2147483647 (inclusive)]
\item {\it Parameter name:} {\tt Temperature polynomial degree}


\index[prmindex]{Temperature polynomial degree}
\index[prmindexfull]{Discretization!Temperature polynomial degree}
{\it Value:} 2


{\it Default:} 2


{\it Description:} The polynomial degree to use for the temperature variable. Units: None.


{\it Possible values:} [Integer range 1...2147483647 (inclusive)]
\item {\it Parameter name:} {\tt Use locally conservative discretization}


\index[prmindex]{Use locally conservative discretization}
\index[prmindexfull]{Discretization!Use locally conservative discretization}
{\it Value:} false


{\it Default:} false


{\it Description:} Whether to use a Stokes discretization that is locally conservative at the expense of a larger number of degrees of freedom (true), or to go with a cheaper discretization that does not locally conserve mass, although it is globally conservative (false).


{\it Possible values:} [Bool]
\end{itemize}



\subsection{Parameters in section \tt Discretization/Stabilization parameters}
\label{parameters:Discretization/Stabilization_20parameters}

\begin{itemize}
\item {\it Parameter name:} {\tt alpha}


\index[prmindex]{alpha}
\index[prmindexfull]{Discretization!Stabilization parameters!alpha}
{\it Value:} 2


{\it Default:} 2


{\it Description:} The exponent $\alpha$ in the entropy viscosity stabilization. Units: None.


{\it Possible values:} [Double 1...2 (inclusive)]
\item {\it Parameter name:} {\tt beta}


\index[prmindex]{beta}
\index[prmindexfull]{Discretization!Stabilization parameters!beta}
{\it Value:} 0.078


{\it Default:} 0.078


{\it Description:} The $\beta$ factor in the artificial viscosity stabilization. An appropriate value for 2d is 0.052 and 0.078 for 3d. Units: None.


{\it Possible values:} [Double 0...1.79769e+308 (inclusive)]
\item {\it Parameter name:} {\tt cR}


\index[prmindex]{cR}
\index[prmindexfull]{Discretization!Stabilization parameters!cR}
{\it Value:} 0.5


{\it Default:} 0.11


{\it Description:} The $c_R$ factor in the entropy viscosity stabilization. Units: None.


{\it Possible values:} [Double 0...1.79769e+308 (inclusive)]
\end{itemize}

\subsection{Parameters in section \tt Geometry model}
\label{parameters:Geometry_20model}

\begin{itemize}
\item {\it Parameter name:} {\tt Model name}


\index[prmindex]{Model name}
\index[prmindexfull]{Geometry model!Model name}
{\it Value:} box


{\it Default:}


{\it Description:} Select one of the following models:

`spherical shell': A geometry representing a spherical shell or a pice of it. Inner and outer radii are read from the parameter file in subsection 'Spherical shell'.

`box': A box geometry parallel to the coordinate directions. The extent of the box in each coordinate direction is set in the parameter file.


{\it Possible values:} [Selection spherical shell|box ]
\end{itemize}



\subsection{Parameters in section \tt Geometry model/Box}
\label{parameters:Geometry_20model/Box}

\begin{itemize}
\item {\it Parameter name:} {\tt X extent}


\index[prmindex]{X extent}
\index[prmindexfull]{Geometry model!Box!X extent}
{\it Value:} 1.2


{\it Default:} 1


{\it Description:} Extent of the box in x-direction. Units: m.


{\it Possible values:} [Double 0...1.79769e+308 (inclusive)]
\item {\it Parameter name:} {\tt Y extent}


\index[prmindex]{Y extent}
\index[prmindexfull]{Geometry model!Box!Y extent}
{\it Value:} 1


{\it Default:} 1


{\it Description:} Extent of the box in y-direction. Units: m.


{\it Possible values:} [Double 0...1.79769e+308 (inclusive)]
\item {\it Parameter name:} {\tt Z extent}


\index[prmindex]{Z extent}
\index[prmindexfull]{Geometry model!Box!Z extent}
{\it Value:} 1


{\it Default:} 1


{\it Description:} Extent of the box in z-direction. This value is ignored if the simulation is in 2d Units: m.


{\it Possible values:} [Double 0...1.79769e+308 (inclusive)]
\end{itemize}

\subsection{Parameters in section \tt Geometry model/Spherical shell}
\label{parameters:Geometry_20model/Spherical_20shell}

\begin{itemize}
\item {\it Parameter name:} {\tt Inner radius}


\index[prmindex]{Inner radius}
\index[prmindexfull]{Geometry model!Spherical shell!Inner radius}
{\it Value:} 3481000


{\it Default:} 3481000


{\it Description:} Inner radius of the spherical shell. Units: m.


{\it Possible values:} [Double 0...1.79769e+308 (inclusive)]
\item {\it Parameter name:} {\tt Opening angle}


\index[prmindex]{Opening angle}
\index[prmindexfull]{Geometry model!Spherical shell!Opening angle}
{\it Value:} 360


{\it Default:} 360


{\it Description:} Opening angle in degrees of the section of the shell that we want to build. Units: degrees.


{\it Possible values:} [Double 0...360 (inclusive)]
\item {\it Parameter name:} {\tt Outer radius}


\index[prmindex]{Outer radius}
\index[prmindexfull]{Geometry model!Spherical shell!Outer radius}
{\it Value:} 6336000


{\it Default:} 6336000


{\it Description:} Outer radius of the spherical shell. Units: m.


{\it Possible values:} [Double 0...1.79769e+308 (inclusive)]
\end{itemize}

\subsection{Parameters in section \tt Gravity model}
\label{parameters:Gravity_20model}

\begin{itemize}
\item {\it Parameter name:} {\tt Model name}


\index[prmindex]{Model name}
\index[prmindexfull]{Gravity model!Model name}
{\it Value:} vertical


{\it Default:}


{\it Description:} Select one of the following models:

`vertical': A gravity model in which the gravity direction is vertically downward and at constant magnitude.

`radial constant': A gravity model in which the gravity direction is radially inward and at constant magnitude. The magnitude is read from the parameter file in subsection 'Radial constant'.

`radial earth-like': A gravity model in which the gravity direction is radially inward and with a magnitude that matches that of the earth at the core-mantle boundary as well as at the surface and in between is physically correct under the assumption of a constant density.


{\it Possible values:} [Selection vertical|radial constant|radial earth-like ]
\end{itemize}



\subsection{Parameters in section \tt Gravity model/Radial constant}
\label{parameters:Gravity_20model/Radial_20constant}

\begin{itemize}
\item {\it Parameter name:} {\tt Magnitude}


\index[prmindex]{Magnitude}
\index[prmindexfull]{Gravity model!Radial constant!Magnitude}
{\it Value:} 30


{\it Default:} 30


{\it Description:} Magnitude of the gravity vector in $m/s^2$. The direction is always radially outward from the center of the earth.


{\it Possible values:} [Double 0...1.79769e+308 (inclusive)]
\end{itemize}

\subsection{Parameters in section \tt Initial conditions}
\label{parameters:Initial_20conditions}

\begin{itemize}
\item {\it Parameter name:} {\tt Model name}


\index[prmindex]{Model name}
\index[prmindexfull]{Initial conditions!Model name}
{\it Value:} perturbed box


{\it Default:}


{\it Description:} Select one of the following models:

`spherical hexagonal perturbation': An initial temperature field in which the temperature is perturbed following a six-fold pattern in angular direction from an otherwise spherically symmetric state.

`spherical gaussian perturbation': An initial temperature field in which the temperature is perturbed by a single Gaussian added to an otherwise spherically symmetric state. Additional parameters are read from the parameter file in subsection 'Spherical gaussian perturbation'.

`perturbed box': An initial temperature field in which the temperature is perturbed slightly from an otherwise constant value equal to one. The perturbation is chosen in such a way that the initial temperature is constant to one along the entire boundary.


{\it Possible values:} [Selection spherical hexagonal perturbation|spherical gaussian perturbation|perturbed box ]
\end{itemize}



\subsection{Parameters in section \tt Initial conditions/Spherical gaussian perturbation}
\label{parameters:Initial_20conditions/Spherical_20gaussian_20perturbation}

\begin{itemize}
\item {\it Parameter name:} {\tt Amplitude}


\index[prmindex]{Amplitude}
\index[prmindexfull]{Initial conditions!Spherical gaussian perturbation!Amplitude}
{\it Value:} 0.01


{\it Default:} 0.01


{\it Description:} The amplitude of the perturbation.


{\it Possible values:} [Double 0...1.79769e+308 (inclusive)]
\item {\it Parameter name:} {\tt Angle}


\index[prmindex]{Angle}
\index[prmindexfull]{Initial conditions!Spherical gaussian perturbation!Angle}
{\it Value:} 0e0


{\it Default:} 0e0


{\it Description:} The angle where the center of the perturbation is placed.


{\it Possible values:} [Double 0...1.79769e+308 (inclusive)]
\item {\it Parameter name:} {\tt Filename for initial geotherm table}


\index[prmindex]{Filename for initial geotherm table}
\index[prmindexfull]{Initial conditions!Spherical gaussian perturbation!Filename for initial geotherm table}
{\it Value:} initial\_geotherm\_table


{\it Default:} initial\_geotherm\_table


{\it Description:} The file from which the initial geotherm table is to be read. The format of the file is defined by what is read in source/initial\_conditions/spherical\_shell.cc.


{\it Possible values:} [FileName (Type: input)]
\item {\it Parameter name:} {\tt Non-dimensional depth}


\index[prmindex]{Non-dimensional depth}
\index[prmindexfull]{Initial conditions!Spherical gaussian perturbation!Non-dimensional depth}
{\it Value:} 0.7


{\it Default:} 0.7


{\it Description:} The non-dimensional radial distance where the center of the perturbation is placed.


{\it Possible values:} [Double 0...1.79769e+308 (inclusive)]
\item {\it Parameter name:} {\tt Sigma}


\index[prmindex]{Sigma}
\index[prmindexfull]{Initial conditions!Spherical gaussian perturbation!Sigma}
{\it Value:} 0.2


{\it Default:} 0.2


{\it Description:} The standard deviation of the Gaussian perturbation.


{\it Possible values:} [Double 0...1.79769e+308 (inclusive)]
\item {\it Parameter name:} {\tt Sign}


\index[prmindex]{Sign}
\index[prmindexfull]{Initial conditions!Spherical gaussian perturbation!Sign}
{\it Value:} 1


{\it Default:} 1


{\it Description:} The sign of the perturbation.


{\it Possible values:} [Double -1.79769e+308...1.79769e+308 (inclusive)]
\end{itemize}

\subsection{Parameters in section \tt Material model}
\label{parameters:Material_20model}

\begin{itemize}
\item {\it Parameter name:} {\tt Model name}


\index[prmindex]{Model name}
\index[prmindexfull]{Material model!Model name}
{\it Value:} simple


{\it Default:}


{\it Description:} Select one of the following models:

`table': A material model that reads tables of pressure and temperature dependent material coefficients from files.

`Steinberger': lookup from the paper of Steinberger/Calderwood

`simple': A simple material model that has constant values for all coefficients but the density. This model uses the formulation that assumes an incompressible medium despite the fact that the density follows the law $\rho(T)=\rho_0(1-\beta(T-T_{\text{ref}})$. The value for the components of this formula and additional parameters are read from the parameter file in subsection 'Simple model'.


{\it Possible values:} [Selection table|Steinberger|simple ]
\end{itemize}



\subsection{Parameters in section \tt Material model/Simple model}
\label{parameters:Material_20model/Simple_20model}

\begin{itemize}
\item {\it Parameter name:} {\tt Reference density}


\index[prmindex]{Reference density}
\index[prmindexfull]{Material model!Simple model!Reference density}
{\it Value:} 1


{\it Default:} 3300


{\it Description:} Reference density $\rho_0$. Units: $kg/m^3$.


{\it Possible values:} [Double 0...1.79769e+308 (inclusive)]
\item {\it Parameter name:} {\tt Reference specific heat}


\index[prmindex]{Reference specific heat}
\index[prmindexfull]{Material model!Simple model!Reference specific heat}
{\it Value:} 1250


{\it Default:} 1250


{\it Description:} The value of the specific heat $cp$. Units: $JG/kgK$.


{\it Possible values:} [Double 0...1.79769e+308 (inclusive)]
\item {\it Parameter name:} {\tt Reference temperature}


\index[prmindex]{Reference temperature}
\index[prmindexfull]{Material model!Simple model!Reference temperature}
{\it Value:} 1


{\it Default:} 293


{\it Description:} The reference temperature $T_0$. Units: $K$.


{\it Possible values:} [Double 0...1.79769e+308 (inclusive)]
\item {\it Parameter name:} {\tt Thermal conductivity}


\index[prmindex]{Thermal conductivity}
\index[prmindexfull]{Material model!Simple model!Thermal conductivity}
{\it Value:} 1e-6


{\it Default:} 4.7


{\it Description:} The value of the thermal conductivity $k$. Units: $W/m/K$.


{\it Possible values:} [Double 0...1.79769e+308 (inclusive)]
\item {\it Parameter name:} {\tt Thermal expansion coefficient}


\index[prmindex]{Thermal expansion coefficient}
\index[prmindexfull]{Material model!Simple model!Thermal expansion coefficient}
{\it Value:} 2e-5


{\it Default:} 2e-5


{\it Description:} The value of the thermal expansion coefficient $\beta$. Units: $1/K$.


{\it Possible values:} [Double 0...1.79769e+308 (inclusive)]
\item {\it Parameter name:} {\tt Viscosity}


\index[prmindex]{Viscosity}
\index[prmindexfull]{Material model!Simple model!Viscosity}
{\it Value:} 1


{\it Default:} 5e24


{\it Description:} The value of the constant viscosity. Units: $kg/m/s$.


{\it Possible values:} [Double 0...1.79769e+308 (inclusive)]
\end{itemize}

\subsection{Parameters in section \tt Material model/Table model}
\label{parameters:Material_20model/Table_20model}

\begin{itemize}
\item {\it Parameter name:} {\tt Composition}


\index[prmindex]{Composition}
\index[prmindexfull]{Material model!Table model!Composition}
{\it Value:} olixene


{\it Default:} olixene


{\it Description:} The Composition of the model.


{\it Possible values:} [Anything]
\item {\it Parameter name:} {\tt Compressible}


\index[prmindex]{Compressible}
\index[prmindexfull]{Material model!Table model!Compressible}
{\it Value:} true


{\it Default:} true


{\it Description:} whether the model is compressible.


{\it Possible values:} [Bool]
\item {\it Parameter name:} {\tt ComputePhases}


\index[prmindex]{ComputePhases}
\index[prmindexfull]{Material model!Table model!ComputePhases}
{\it Value:} false


{\it Default:} false


{\it Description:} whether to compute phases.


{\it Possible values:} [Bool]
\item {\it Parameter name:} {\tt Gravity}


\index[prmindex]{Gravity}
\index[prmindexfull]{Material model!Table model!Gravity}
{\it Value:} 30


{\it Default:} 30


{\it Description:} The value of the gravity constant.Units: $m/s^2$.


{\it Possible values:} [Double 0...1.79769e+308 (inclusive)]
\item {\it Parameter name:} {\tt Path to model data}


\index[prmindex]{Path to model data}
\index[prmindexfull]{Material model!Table model!Path to model data}
{\it Value:} datadir


{\it Default:} datadir


{\it Description:} The path to the model data.


{\it Possible values:} [DirectoryName]
\item {\it Parameter name:} {\tt Reference density}


\index[prmindex]{Reference density}
\index[prmindexfull]{Material model!Table model!Reference density}
{\it Value:} 3300


{\it Default:} 3300


{\it Description:} Reference density $\rho_0$. Units: $kg/m^3$.


{\it Possible values:} [Double 0...1.79769e+308 (inclusive)]
\item {\it Parameter name:} {\tt Reference specific heat}


\index[prmindex]{Reference specific heat}
\index[prmindexfull]{Material model!Table model!Reference specific heat}
{\it Value:} 1250


{\it Default:} 1250


{\it Description:} The value of the specific heat $cp$. Units: $JG/kgK$.


{\it Possible values:} [Double 0...1.79769e+308 (inclusive)]
\item {\it Parameter name:} {\tt Reference temperature}


\index[prmindex]{Reference temperature}
\index[prmindexfull]{Material model!Table model!Reference temperature}
{\it Value:} 293


{\it Default:} 293


{\it Description:} The reference temperature $T_0$. Units: $K$.


{\it Possible values:} [Double 0...1.79769e+308 (inclusive)]
\item {\it Parameter name:} {\tt Thermal conductivity}


\index[prmindex]{Thermal conductivity}
\index[prmindexfull]{Material model!Table model!Thermal conductivity}
{\it Value:} 4.7


{\it Default:} 4.7


{\it Description:} The value of the thermal conductivity $k$. Units: $W/m/K$.


{\it Possible values:} [Double 0...1.79769e+308 (inclusive)]
\item {\it Parameter name:} {\tt Thermal expansion coefficient}


\index[prmindex]{Thermal expansion coefficient}
\index[prmindexfull]{Material model!Table model!Thermal expansion coefficient}
{\it Value:} 2e-5


{\it Default:} 2e-5


{\it Description:} The value of the thermal expansion coefficient $\beta$. Units: $1/K$.


{\it Possible values:} [Double 0...1.79769e+308 (inclusive)]
\end{itemize}



\subsection{Parameters in section \tt Material model/Table model/Viscosity}
\label{parameters:Material_20model/Table_20model/Viscosity}

\begin{itemize}
\item {\it Parameter name:} {\tt ReferenceViscosity}


\index[prmindex]{ReferenceViscosity}
\index[prmindexfull]{Material model!Table model!Viscosity!ReferenceViscosity}
{\it Value:} 5e24


{\it Default:} 5e24


{\it Description:} The value of the constant viscosity. Units: $kg/m/s$.


{\it Possible values:} [Double 0...1.79769e+308 (inclusive)]
\item {\it Parameter name:} {\tt ViscosityModel}


\index[prmindex]{ViscosityModel}
\index[prmindexfull]{Material model!Table model!Viscosity!ViscosityModel}
{\it Value:} Exponential


{\it Default:} Exponential


{\it Description:} Viscosity Model


{\it Possible values:} [Anything]
\item {\it Parameter name:} {\tt Viscosity increase lower mantle}


\index[prmindex]{Viscosity increase lower mantle}
\index[prmindexfull]{Material model!Table model!Viscosity!Viscosity increase lower mantle}
{\it Value:} 4e0


{\it Default:} 4e0


{\it Description:} The Viscosity increase (jump) in the lower mantle.


{\it Possible values:} [Double 0...1.79769e+308 (inclusive)]
\item {\it Parameter name:} {\tt exponential\_P}


\index[prmindex]{exponential\_P}
\index[prmindexfull]{Material model!Table model!Viscosity!exponential\_P}
{\it Value:} 1


{\it Default:} 1


{\it Description:} multiplication factor or Pressure exponent


{\it Possible values:} [Double 0...1.79769e+308 (inclusive)]
\item {\it Parameter name:} {\tt exponential\_T}


\index[prmindex]{exponential\_T}
\index[prmindexfull]{Material model!Table model!Viscosity!exponential\_T}
{\it Value:} 1


{\it Default:} 1


{\it Description:} multiplication factor or Temperature exponent


{\it Possible values:} [Double 0...1.79769e+308 (inclusive)]
\end{itemize}

\subsection{Parameters in section \tt Mesh refinement}
\label{parameters:Mesh_20refinement}

\begin{itemize}
\item {\it Parameter name:} {\tt Additional refinement times}


\index[prmindex]{Additional refinement times}
\index[prmindexfull]{Mesh refinement!Additional refinement times}
{\it Value:}


{\it Default:}


{\it Description:} A list of times so that if the end time of a time step is beyond this time, an additional round of mesh refinement is triggered. This is mostly useful to make sure we can get through the initial transient phase of a simulation on a relatively coarse mesh, and then refine again when we are in a time range that we are interested in and where we would like to use a finer mesh. Units: each element of the list has units years if the 'Use years in output instead of seconds' parameter is set; seconds otherwise.


{\it Possible values:} [List list of <[Double 0...1.79769e+308 (inclusive)]> of length 0...4294967295 (inclusive)]
\item {\it Parameter name:} {\tt Coarsening fraction}


\index[prmindex]{Coarsening fraction}
\index[prmindexfull]{Mesh refinement!Coarsening fraction}
{\it Value:} 0.05


{\it Default:} 0.05


{\it Description:} The fraction of cells with the smallest error that should be flagged for coarsening.


{\it Possible values:} [Double 0...1 (inclusive)]
\item {\it Parameter name:} {\tt Initial adaptive refinement}


\index[prmindex]{Initial adaptive refinement}
\index[prmindexfull]{Mesh refinement!Initial adaptive refinement}
{\it Value:} 1


{\it Default:} 2


{\it Description:} The number of adaptive refinement steps performed after initial global refinement but while still within the first time step.


{\it Possible values:} [Integer range 0...2147483647 (inclusive)]
\item {\it Parameter name:} {\tt Initial global refinement}


\index[prmindex]{Initial global refinement}
\index[prmindexfull]{Mesh refinement!Initial global refinement}
{\it Value:} 4


{\it Default:} 2


{\it Description:} The number of global refinement steps performed on the initial coarse mesh, before the problem is first solved there.


{\it Possible values:} [Integer range 0...2147483647 (inclusive)]
\item {\it Parameter name:} {\tt Refinement fraction}


\index[prmindex]{Refinement fraction}
\index[prmindexfull]{Mesh refinement!Refinement fraction}
{\it Value:} 0.3


{\it Default:} 0.3


{\it Description:} The fraction of cells with the largest error that should be flagged for refinement.


{\it Possible values:} [Double 0...1 (inclusive)]
\item {\it Parameter name:} {\tt Strategy}


\index[prmindex]{Strategy}
\index[prmindexfull]{Mesh refinement!Strategy}
{\it Value:} Density c\_p temperature


{\it Default:} Density c\_p temperature


{\it Description:} The method used to determine which cells to refine and which to coarsen.


{\it Possible values:} [Selection Temperature|Normalized density and temperature|Weighted density and temperature|Density c\_p temperature ]
\item {\it Parameter name:} {\tt Time steps between mesh refinement}


\index[prmindex]{Time steps between mesh refinement}
\index[prmindexfull]{Mesh refinement!Time steps between mesh refinement}
{\it Value:} 5


{\it Default:} 10


{\it Description:} The number of time steps after which the mesh is to be adapted again based on computed error indicators.


{\it Possible values:} [Integer range 1...2147483647 (inclusive)]
\end{itemize}

\subsection{Parameters in section \tt Model settings}
\label{parameters:Model_20settings}

\begin{itemize}
\item {\it Parameter name:} {\tt Fixed temperature boundary indicators}


\index[prmindex]{Fixed temperature boundary indicators}
\index[prmindexfull]{Model settings!Fixed temperature boundary indicators}
{\it Value:} 0, 1


{\it Default:}


{\it Description:} A comma separated list of integers denoting those boundaries on which the temperature is fixed and described by the boundary temperature object selected in its own section of this input file. All boundary indicators used by the geometry but not explicitly listed here will end up with no-flux (insulating) boundary conditions.


{\it Possible values:} [List list of <[Integer range 0...2147483647 (inclusive)]> of length 0...4294967295 (inclusive)]
\item {\it Parameter name:} {\tt Include adiabatic heating}


\index[prmindex]{Include adiabatic heating}
\index[prmindexfull]{Model settings!Include adiabatic heating}
{\it Value:} false


{\it Default:} false


{\it Description:} Whether to include adiabatic heating into the model or not. From a physical viewpoint, adiabatic heating should always be used but may be undesirable when comparing results with known benchmarks that do not include this term in the temperature equation.


{\it Possible values:} [Bool]
\item {\it Parameter name:} {\tt Include shear heating}


\index[prmindex]{Include shear heating}
\index[prmindexfull]{Model settings!Include shear heating}
{\it Value:} false


{\it Default:} true


{\it Description:} Whether to include shear heating into the model or not. From a physical viewpoint, shear heating should always be used but may be undesirable when comparing results with known benchmarks that do not include this term in the temperature equation.


{\it Possible values:} [Bool]
\item {\it Parameter name:} {\tt Prescribed velocity boundary indicators}


\index[prmindex]{Prescribed velocity boundary indicators}
\index[prmindexfull]{Model settings!Prescribed velocity boundary indicators}
{\it Value:} 0


{\it Default:}


{\it Description:} A comma separated list of integers denoting those boundaries on which the velocity is tangential but prescribed, i.e., where external forces act to prescribe a particular velocity. This is often used to prescribe a velocity that equals that of overlying plates.


{\it Possible values:} [List list of <[Integer range 0...255 (inclusive)]> of length 0...4294967295 (inclusive)]
\item {\it Parameter name:} {\tt Radiogenic heating rate}


\index[prmindex]{Radiogenic heating rate}
\index[prmindexfull]{Model settings!Radiogenic heating rate}
{\it Value:} 0


{\it Default:} 0e0


{\it Description:} H0


{\it Possible values:} [Double -1.79769e+308...1.79769e+308 (inclusive)]
\item {\it Parameter name:} {\tt Tangential velocity boundary indicators}


\index[prmindex]{Tangential velocity boundary indicators}
\index[prmindexfull]{Model settings!Tangential velocity boundary indicators}
{\it Value:} 1


{\it Default:}


{\it Description:} A comma separated list of integers denoting those boundaries on which the velocity is tangential and unrestrained, i.e., where no external forces act to prescribe a particular tangential velocity (although there is a force that requires the flow to be tangential).


{\it Possible values:} [List list of <[Integer range 0...255 (inclusive)]> of length 0...4294967295 (inclusive)]
\item {\it Parameter name:} {\tt Zero velocity boundary indicators}


\index[prmindex]{Zero velocity boundary indicators}
\index[prmindexfull]{Model settings!Zero velocity boundary indicators}
{\it Value:} 2, 3


{\it Default:}


{\it Description:} A comma separated list of integers denoting those boundaries on which the velocity is zero.


{\it Possible values:} [List list of <[Integer range 0...255 (inclusive)]> of length 0...4294967295 (inclusive)]
\end{itemize}

\subsection{Parameters in section \tt Postprocess}
\label{parameters:Postprocess}

\begin{itemize}
\item {\it Parameter name:} {\tt List of postprocessors}


\index[prmindex]{List of postprocessors}
\index[prmindexfull]{Postprocess!List of postprocessors}
{\it Value:} visualization, velocity statistics, temperature statistics, heat flux statistics


{\it Default:} all


{\it Description:} A comma separated list of postprocessor objects that should be run at the end of each time step. Some of these postprocessors will declare their own parameters which may, for example, include that they will actually do something only every so many time steps or years. Alternatively, the text 'all' indicates that all available postprocessors should be run after each time step.

The following postprocessors are available:

`visualization': A postprocessor that takes the solution and writes it into files that can be read by a graphical visualization program. Additional run time parameters are read from the parameter subsection 'Visualization'.

`velocity statistics': A postprocessor that computes some statistics about the velocity field.

`temperature statistics': A postprocessor that computes some statistics about the temperature field.

`heat flux statistics': A postprocessor that computes some statistics about the heat flux across boundaries.

`depth average': A postprocessor that computes depth averaged quantities and writes them out.


{\it Possible values:} [MultipleSelection visualization|velocity statistics|temperature statistics|heat flux statistics|depth average|all ]
\end{itemize}



\subsection{Parameters in section \tt Postprocess/Depth average}
\label{parameters:Postprocess/Depth_20average}

\begin{itemize}
\item {\it Parameter name:} {\tt Time between graphical output}


\index[prmindex]{Time between graphical output}
\index[prmindexfull]{Postprocess!Depth average!Time between graphical output}
{\it Value:} 1e8


{\it Default:} 1e8


{\it Description:} The time interval between each generation of graphical output files. A value of zero indicates that output should be generated in each time step. Units: years if the 'Use years in output instead of seconds' parameter is set; seconds otherwise.


{\it Possible values:} [Double 0...1.79769e+308 (inclusive)]
\end{itemize}

\subsection{Parameters in section \tt Postprocess/Visualization}
\label{parameters:Postprocess/Visualization}

\begin{itemize}
\item {\it Parameter name:} {\tt Number of grouped files}


\index[prmindex]{Number of grouped files}
\index[prmindexfull]{Postprocess!Visualization!Number of grouped files}
{\it Value:} 0


{\it Default:} 0


{\it Description:} VTU file output supports grouping files from several CPUs into one file using MPI I/O when writing on a parallel filesystem. Select 0 for no grouping. This will disable parallel file output and instead write one file per processor in a background thread. A value of 1 will generate one big file containing the whole solution.


{\it Possible values:} [Integer range 0...2147483647 (inclusive)]
\item {\it Parameter name:} {\tt Output format}


\index[prmindex]{Output format}
\index[prmindexfull]{Postprocess!Visualization!Output format}
{\it Value:} vtu


{\it Default:} vtu


{\it Description:} The file format to be used for graphical output.


{\it Possible values:} [Selection none|dx|ucd|gnuplot|povray|eps|gmv|tecplot|tecplot\_binary|vtk|vtu|deal.II intermediate ]
\item {\it Parameter name:} {\tt Time between graphical output}


\index[prmindex]{Time between graphical output}
\index[prmindexfull]{Postprocess!Visualization!Time between graphical output}
{\it Value:} 0


{\it Default:} 1e8


{\it Description:} The time interval between each generation of graphical output files. A value of zero indicates that output should be generated in each time step. Units: years if the 'Use years in output instead of seconds' parameter is set; seconds otherwise.


{\it Possible values:} [Double 0...1.79769e+308 (inclusive)]
\end{itemize}
